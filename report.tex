
\documentclass[12pt,letterpaper]{article}

\usepackage[english]{babel}
\usepackage[utf8]{inputenc}
\usepackage[T1]{fontenc}
\usepackage{fullpage}
\usepackage{cancel}
\usepackage{booktabs}
\usepackage[top=2cm, bottom=4.5cm, left=2.5cm, right=2.5cm]{geometry}
\usepackage{amsmath,amsthm,amsfonts,amssymb,amscd}
\usepackage{lastpage}
\usepackage{enumerate}
\usepackage{fancyhdr}
\usepackage{mathrsfs}
\usepackage{xcolor}
\usepackage{graphicx}
\usepackage{listings}
\usepackage{hyperref}
\usepackage{tikz}
\usepackage{tikz-cd}
\usepackage{pgfplots}
\usepackage[
backend=bibtex,
sorting=ynt
]{biblatex}
\addbibresource{refs.bib}

\newtheorem{defi}{Definition}
\newtheorem{teo}{Theorem}
\newtheorem{prop}{Proposition}
\hypersetup{%
	colorlinks=true,
	linkcolor=blue,
	linkbordercolor={0 0 1}
}

\setlength{\parindent}{0.0in}
\setlength{\parskip}{0.05in}

\newcommand\course{Rener Oliveira}
\newcommand\lcur{\mathcal{L}}
\newcommand{\real}{\mathbb{R}}
\newcommand{\rr}{\mathbb{R}^2}
\newcommand{\rn}{\mathbb{R}^n}
\newcommand{\linesep}{{\color{black} \rule{\linewidth}{0.5mm} }}
\newcommand{\rpos}{\mathbb{R}_{>0}}
\newcommand{\blue}[1]{{\color{blue}{#1}}}
\newcommand{\bd}[1]{\boldsymbol{#1}}
\newcommand{\gt}{>}%broken keyboard
\newcommand{\pow}{^}%broken keyboard
\newcommand{\pr}{\operatorname{Pr}} %% probability
\newcommand{\vr}{\operatorname{Var}} %% variance
\newcommand{\rs}{X_1, X_2, \ldots, X_p} %%  random sample
\newcommand{\irs}{X_1, X_2, \ldots} %% infinite random sample
\newcommand{\rsd}{x_1, x_2, \ldots, x_p} %%  random sample, realised
\newcommand{\Sm}{\bar{X}_n} %%  sample mean, random variable
\newcommand{\sm}{\bar{x}_n} %%  sample mean, realised
\newcommand{\Sv}{\bar{S}^2_n} %%  sample variance, random variable
\newcommand{\sv}{\bar{s}^2_n} %%  sample variance, realised
\newcommand{\bX}{\boldsymbol{X}} %%  random sample, contracted form (bold)
\newcommand{\bx}{\boldsymbol{x}} %%  random sample, realised, contracted form (bold)
\newcommand{\bT}{\boldsymbol{T}} %%  Statistic, vector form (bold)
\newcommand{\bt}{\boldsymbol{t}} %%  Statistic, realised, vector form (bold)
\newcommand{\emv}{\hat{\theta}_{\text{EMV}}}
\newcommand{\defn}{\stackrel{\textrm{\scriptsize def}}{=}}
\newcommand{\op}{\operatorname}
\newcommand{\eps}{\varepsilon}
\newcommand{\norm}{\mathcal{N}}
\newcommand{\N}{\mathbb{N}}
\newcommand{\iid}{\overset{\text{iid}}{\sim}}
\pagestyle{fancyplain}
\headheight 35pt        
\chead{\textbf{\Large Churn Prediction Project}}
\lhead{Machine Learning\\EMAp FGV}
\rhead{\small{\course \\ \today}}
\lfoot{}
\cfoot{}
\rfoot{\small\thepage}
\headsep 1.5em
\usepackage{xcolor}
\definecolor{maroon}{cmyk}{0, 0.87, 0.68, 0.32}
\definecolor{halfgray}{gray}{0.55}
\definecolor{ipython_frame}{RGB}{207, 207, 207}
\definecolor{ipython_bg}{RGB}{247, 247, 247}
\definecolor{ipython_red}{RGB}{186, 33, 33}
\definecolor{ipython_green}{RGB}{0, 128, 0}
\definecolor{ipython_cyan}{RGB}{64, 128, 128}
\definecolor{ipython_purple}{RGB}{170, 34, 255}

\usepackage{listings}
\usepackage{color}
\definecolor{dkgreen}{rgb}{0,0.6,0}
\definecolor{gray}{rgb}{0.5,0.5,0.5}
\definecolor{mauve}{rgb}{0.58,0,0.82}
\lstset{ %
	language=R,                     % the language of the code
	basicstyle=\footnotesize,       % the size of the fonts that are used for the code
	numbers=left,                   % where to put the line-numbers
	numberstyle=\tiny\color{gray},  % the style that is used for the line-numbers
	stepnumber=1,                   % the step between two line-numbers. If it's 1, each line
	% will be numbered
	numbersep=5pt,                  % how far the line-numbers are from the code
	backgroundcolor=\color{white},  % choose the background color. You must add \usepackage{color}
	showspaces=false,               % show spaces adding particular underscores
	showstringspaces=false,         % underline spaces within strings
	showtabs=false,                 % show tabs within strings adding particular underscores
	frame=single,                   % adds a frame around the code
	rulecolor=\color{black},        % if not set, the frame-color may be changed on line-breaks within not-black text (e.g. commens (green here))
	tabsize=2,                      % sets default tabsize to 2 spaces
	captionpos=b,                   % sets the caption-position to bottom
	breaklines=true,                % sets automatic line breaking
	breakatwhitespace=false,        % sets if automatic breaks should only happen at whitespace
	title=\lstname,                 % show the filename of files included with \lstinputlisting;
	% also try caption instead of title
	keywordstyle=\color{blue},      % keyword style
	commentstyle=\color{dkgreen},   % comment style
	stringstyle=\color{mauve},      % string literal style
	%escapeinside={\%*}{*)},         % if you want to add a comment within your code
	morekeywords={*,...}            % if you want to add more keywords to the set
} 
\lstset{
	breaklines=true,
	%
	extendedchars=true,
	literate=
	{á}{{\'a}}1 {é}{{\'e}}1 {í}{{\'i}}1 {ó}{{\'o}}1 {ú}{{\'u}}1
	{Á}{{\'A}}1 {É}{{\'E}}1 {Í}{{\'I}}1 {Ó}{{\'O}}1 {Ú}{{\'U}}1
	{à}{{\`a}}1 {è}{{\`e}}1 {ì}{{\`i}}1 {ò}{{\`o}}1 {ù}{{\`u}}1
	{À}{{\`A}}1 {È}{{\'E}}1 {Ì}{{\`I}}1 {Ò}{{\`O}}1 {Ù}{{\`U}}1
	{ä}{{\"a}}1 {ë}{{\"e}}1 {ï}{{\"i}}1 {ö}{{\"o}}1 {ü}{{\"u}}1
	{Ä}{{\"A}}1 {Ë}{{\"E}}1 {Ï}{{\"I}}1 {Ö}{{\"O}}1 {Ü}{{\"U}}1
	{â}{{\^a}}1 {ê}{{\^e}}1 {î}{{\^i}}1 {ô}{{\^o}}1 {û}{{\^u}}1
	{Â}{{\^A}}1 {Ê}{{\^E}}1 {Î}{{\^I}}1 {Ô}{{\^O}}1 {Û}{{\^U}}1
	{œ}{{\oe}}1 {Œ}{{\OE}}1 {æ}{{\ae}}1 {Æ}{{\AE}}1 {ß}{{\ss}}1
	{ç}{{\c c}}1 {Ç}{{\c C}}1 {ø}{{\o}}1 {å}{{\r a}}1 {Å}{{\r A}}1
	{€}{{\EUR}}1 {£}{{\pounds}}1
}

%%
%% Python definition (c) 1998 Michael Weber
%% Additional definitions (2013) Alexis Dimitriadis
%% modified by me (should not have empty lines)
%%

\begin{document}
	\tableofcontents
	\newpage
	\section{Introduction}
	
	Customer churn (loss of customers) is a problem for telecommunication companies, considering an environment of increasing competition. When a company loses customers, it not only loses the future revenue, but also the investment made to get those customers. According to \cite{class_imbalance}, some studies show that acquiring new clients is five to six times more expensive than retaining existing ones. 
	
	Telecom companies have two approaches to deal with churners: the reactive one, which tries to convince customers who wants to cancel to stay; and the proactive one, which predicts who is more likely to churn before they explicitly decide to churn and send them suitable offers to avoid their loss.
	
	The aim of this project is to study the churn problem using the \href{https://www.crowdanalytix.com/contests/why-customer-churn/}{CrowdAnalytix} dataset, which has information regarding usage patterns from customers of a telecom company, which the real name was anonymized. This data was part of a Machine Learning Challenge issued by CrowdAnalytix in 2012.
	
	The raw data contains 3333 observations of 20 variables which are described in Table \ref{features}. The \href{https://www.kaggle.com/mnassrib/telecom-churn-datasets}{Kaggle} copy of the data was already divided into 80\% train and 20\% test.
	
	\begin{table}[!htb]
		\centering
		\caption{Features Description}
		\label{features}
		\begin{tabular}{|c|c|} \hline
			\textbf{Variable} & \textbf{Description}\\  \hline 
			 State  &  Customer's state \\ \hline
			 Account.length& Time since subscription \\ \hline
			 Area.code   & Phone number area code\\ \hline 
			 International.plan& Has an international plan? (Yes=1,No=0)\\ \hline 
			 Voice.mail.plan    & Has a voicemail plan? (Yes=1,No=0)\\ \hline
			 Number.vmail.messages& Number of voicemail messages\\ \hline
			 Total.day.minutes& Total minutes used during the day\\ \hline
			 Total.day.calls& Total calls made during days\\ \hline
			 Total.day.charge& Total charge during days\\ \hline
			 Total.eve.minutes& Total minutes used during evenings\\ \hline
			 Total.eve.calls& Total calls made during evenings\\ \hline
			 Total.eve.charge& Total charge during evenings\\ \hline 
			 Total.night.minutes & Total minutes used during nights\\ \hline 
			 Total.night.calls& Total calls made during nights\\ \hline
			 Total.night.charge& Total charge during nights\\ \hline 
			 Total.intl.minutes& Total international minutes used\\ \hline 
			 Total.intl.calls& Total international calls made\\ \hline 
			 Total.intl.charge& Total international charge\\\hline 
			 Customer.service.calls& Number of calls to customer services\\ \hline 
			 Churn & Has the customer churned? (Yes=1,No=0)\\ \hline
		\end{tabular}
	\end{table}
	
	\section{Data Cleaning and EDA}
	
	\section{Classification Methods}
	.... Describe evaluation metrics
	
	.... breafly describe ML methods (Knn, logistic regression, naive bayes (maybe) bagging, random forest, SVM) 
	
	.... describe resampling methods to handle imbalance
	
	.... algorithm-level solutions (cost-sensitive learning,...)
	
	\section{Results}
	present and discuss metric's table ...
	
	\section{Conclusions and Future Work}
	\newpage
	\addcontentsline{toc}{section}{References}
	%			\bibliography{refs}
	\printbibliography
\end{document}